\documentclass{article}

\usepackage[utf8]{inputenc}
\usepackage[T1]{fontenc}
\usepackage[ngerman]{babel}
\usepackage{amsmath, amssymb}
\usepackage{array}
\usepackage{caption}
\usepackage{xcolor}

% --- KOPFZEILE HINZUFÜGEN (Ihr Name) ---
\usepackage{fancyhdr}
\pagestyle{fancy}
\fancyhead{} % Löscht alte Kopfzeilen
\fancyfoot{} % Löscht alte Fußzeilen
\fancyhead[L]{EIMA Übung 0: Florian Kainrath} % ERSETZEN SIE DIES MIT IHREM NAMEN
\fancyhead[R]{\thepage} % Seitenzahl rechts
\renewcommand{\headrulewidth}{0.4pt}
% ---------------------------------------

\begin{document}

\section*{Aufgabe 1}

\begin{table}[h!]
	\centering
	\caption{
		Wahrheitstabelle zur Tautologie
		$\smash{\mbox{$\neg (A \land B) \Leftrightarrow \neg A \lor \neg B$}}$
		und
		$\smash{\mbox{$\neg (A \lor B) \Leftrightarrow \neg A \land \neg B$}}$
	}
	\label{tab:wahrheitstabelle}

	\renewcommand{\arraystretch}{1.2}
	\begin{tabular}{|c|c||c|c||c|c|c|c||c|c|}
		\hline
		$A$ & $B$ & $A \land B$ & $\neg (A \land B)$ & $\neg A$ & $\neg B$ & $\neg A \lor \neg B$ & $A \lor B$ & $\neg (A \lor B)$ & $\neg A \land \neg B$ \\
		\hline\hline
		w   & w   & w           & f                  & f        & f        & f                    & w          & f                 & f                     \\
		\hline
		w   & f   & f           & w                  & f        & w        & w                    & w          & f                 & f                     \\
		\hline
		f   & w   & f           & w                  & w        & f        & w                    & w          & f                 & f                     \\
		\hline
		f   & f   & f           & w                  & w        & w        & w                    & f          & w                 & w                     \\
		\hline
	\end{tabular}
\end{table}
\section*{Aufgabe 4}

Beweisen Sie mittels vollständiger Induktion, dass
\[
	\sum_{k=1}^{n} k^2 = 1^2 + 2^2 + \dots + n^2 = \frac{n(n+1)(2n+1)}{6}
	\quad \text{für alle } n \in \mathbb{N}.
\]

\subsection*{Beweis durch vollständige Induktion}

\textbf{Induktionsanfang:} Für \( n = 1 \) gilt:

\[
	\sum_{k=1}^{1} k^2 = 1^2 = 1
	\quad \text{und} \quad
	\frac{1 \cdot (1 + 1) \cdot (2 \cdot 1 + 1)}{6} = \frac{1 \cdot 2 \cdot 3}{6} = 1
\]

\(\Rightarrow\) Aussage stimmt für \( n = 1 \).

\bigskip

\textbf{Induktionsvoraussetzung:} Sei die Aussage wahr für ein festes, aber beliebiges \( n \in \mathbb{N} \), also:

\[
	\sum_{k=1}^{n} k^2 = \frac{n(n+1)(2n+1)}{6}
\]

\textbf{Induktionsschritt:} Zeige die Aussage für \( n+1 \):

\begin{align*}
	\sum_{k=1}^{n+1} k^2
	 & = \left( \sum_{k=1}^{n} k^2 \right) + (n+1)^2                         \\
	 & = \frac{n(n+1)(2n+1)}{6} + (n+1)^2                                    \\
	 & = \frac{n(n+1)(2n+1) + 6(n+1)^2}{6}                                   \\
	 & = \frac{(n+1)\left[n(2n+1) + 6(n+1)\right]}{6}                        \\
	 & = \frac{(n+1)\left[2n^2 + n + 6n + 6\right]}{6}                       \\
	 & = \frac{(n+1)(2n^2 + 7n + 6)}{6}                                      \\
	 & \quad \textit{(Nebenrechnung: } (n+2)(2n+3) = 2n^2 + 7n + 6\textit{)} \\
	 & = \frac{(n+1)(n+2)(2n+3)}{6}
\end{align*}

\(\Rightarrow\) Aussage stimmt auch für \( n = 1 \).

\bigskip

\textbf{Schluss:} Nach dem Prinzip der vollständigen Induktion gilt:

\[
	\sum_{k=1}^{n} k^2 = \frac{n(n+1)(2n+1)}{6} \quad \text{für alle } n \in \mathbb{N}.
\]

\section*{Aufgabe 5}

Beweisen Sie mittels vollständiger Induktion, dass
\[
	4n < 2^n \quad \text{für alle } n \in \mathbb{N} \text{ mit } n \geq 5 \text{ gilt.}
\]


\subsection*{Beweis durch vollständige Induktion}

\textbf{I. Induktionsanfang ($n=5$):}
Da die Aussage nur für $n \geq 5$ gelten soll, wählen wir den kleinsten Wert, $n_0 = 5$.
\[
	4n < 2^n
\]
\[
	4 * 5 < 2^5
\]
\[
	20 < 32
\]
Da $20 < 32$, ist die Aussage $4 \cdot 5 < 2^5$ wahr.
\(\Rightarrow\) Aussage stimmt für \( n = 5 \).

\bigskip

\textbf{II. Induktionsvoraussetzung (IV):}
Sei die Aussage wahr für ein festes, aber beliebiges \( n \in \mathbb{N} \) mit $n \geq 5$:
\[
	4n < 2^n
\]

\textbf{III. Induktionsschritt (IS):}
Zeige die Aussage für \( n+1 \), d.h., es muss gezeigt werden: $\mathbf{4(n+1) < 2^{n+1}}$.

\begin{align*}
	4(n+1)      & = 4n + 4                                                          \\
	4n + 4      & < 2^n + 4                                                         \\
	2^n  +4     & <  2^n*2               \text{ (nach IV: } 4n < 2^n)               \\
	4n  +4      & \leq 2^n+4 <  2^n*2                                               \\
	\\
	\text{Beweis: }                                                                 \\
	2^n+4       & > 2^n*2                                                           \\
	2^n+4       & > 2^{n+1}                                                         \\
	2^n-2^{n+1} & > 4              \text{ (umgeformt, sodass 4 kleiner sein muss) } \\
	2^n*(2-1)   & > 4                                                               \\
	2^n*1       & > 4 => n = 3                                                      \\
\end{align*}

woraus folgt:
\(\Rightarrow\) Aussage $A(n+1)$ stimmt, wenn n \(\geq \) 5.

\bigskip

\textbf{Schluss:}
Nach dem Prinzip der vollständigen Induktion gilt die Aussage:
\[
	4n < 2^n \quad \text{für alle } n \in \mathbb{N} \text{ mit } n \geq 5.
\]

\section*{Aufgabe 6}

Beweisen Sie mittels vollständiger Induktion, dass für jedes $n \in \mathbb{N}$ mit $n \geq 12$ geeignete $a, b \in \mathbb{N}_0$ gibt, so dass
\[
	n = 4a + 5b \quad \text{gilt.}
\]

\subsection*{Beweis durch vollständige Induktion}

Dieser Beweis erfordert die starke Form der Induktion, da wir im Induktionsschritt die Behauptung für eine Zahl $n-k$ benötigen, wobei $k > 1$. Alternativ können wir mehrere Induktionsanfänge verwenden.

\textbf{I. Induktionsanfang (IA):} Wir zeigen die Aussage für die ersten vier Werte $n=12, 13, 14, 15$.
\begin{itemize}
	\item \textbf{$n=12$:} $12 = 4 \cdot 3 + 5 \cdot 0 \quad (\text{mit } a=3, b=0)$
	\item \textbf{$n=13$:} $13 = 4 \cdot 2 + 5 \cdot 1 \quad (\text{mit } a=2, b=1)$
	\item \textbf{$n=14$:} $14 = 4 \cdot 1 + 5 \cdot 2 \quad (\text{mit } a=1, b=2)$
	\item \textbf{$n=15$:} $15 = 4 \cdot 0 + 5 \cdot 3 \quad (\text{mit } a=0, b=3)$
\end{itemize}
\(\Rightarrow\) Die Aussage gilt für $n=12, 13, 14, 15$.

\bigskip

\textbf{II. Induktionsvoraussetzung (IV):}
Sei $n \geq 15$ beliebig. Die Aussage sei wahr für alle $k \in \{12, 13, \dots, n\}$, d.h., jede Zahl $k$ kann als $k = 4a' + 5b'$ mit $a', b' \in \mathbb{N}_0$ dargestellt werden.

\bigskip

\textbf{III. Induktionsschritt (IS):}
Wir zeigen die Aussage für $n+1$. Es muss gezeigt werden, dass $n+1 = 4a + 5b$ dargestellt werden kann.

Da $n \geq 15$ ist, ist die Zahl $n+1-4 = n-3$ Element der Menge $\{12, 13, \dots, n\}$.
Daher können wir die Induktionsvoraussetzung auf die Zahl $n-3$ anwenden:
\[
	n-3 = 4a' + 5b' \quad \text{für geeignete } a', b' \in \mathbb{N}_0 \quad \text{(nach IV)}
\]

Nun addieren wir 4 auf beiden Seiten der Gleichung:
\begin{align*}
	n+1 & = (n-3) + 4       \\
	    & = (4a' + 5b') + 4 \\
	    & = 4a' + 4 + 5b'   \\
	    & = 4(a' + 1) + 5b'
\end{align*}

Setzen wir $a = a' + 1$ und $b = b'$, so haben wir die Darstellung
\[
	n+1 = 4a + 5b
\]
Da $a' \geq 0$ ist, ist auch $a = a' + 1 \geq 1$, und da $b' \geq 0$ ist, ist $b \geq 0$. Somit sind $a, b \in \mathbb{N}_0$.

\(\Rightarrow\) Aussage stimmt für \( n+1 \).

\bigskip

\textbf{Schluss:}
Nach dem Prinzip der vollständigen Induktion gilt, dass für jedes $n \in \mathbb{N}$ mit $n \geq 12$ geeignete $a, b \in \mathbb{N}_0$ gibt, so dass $n = 4a + 5b$ gilt.

\end{document}
