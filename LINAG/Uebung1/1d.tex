\documentclass{article}

\usepackage[utf8]{inputenc}
\usepackage[T1]{fontenc}
\usepackage[ngerman]{babel}
\usepackage{amsmath, amssymb}
\usepackage{xcolor}
\usepackage{enumitem}
\usepackage{fancyhdr}

% Kopf- und Fußzeilen
\pagestyle{fancy}
\fancyhead{}
\fancyfoot{}
\fancyhead[R]{\thepage}
\renewcommand{\headrulewidth}{0.4pt}

\begin{document}

\section*{Aufgabe 1:d}
Sei \( f : X \to Y \) eine Funktion, \( A \subseteq X \) und
\[
    B := \{ f(a) : a \in A \} = f(A).
\]
Welche der folgenden Aussagen müssen dann gelten?

\begin{enumerate}[label=(\roman*)]
    \item \( \forall x \in X : (x \in A \Rightarrow f(x) \in B) \)
    \item \( \forall x \in X : (f(x) \in B \Rightarrow x \in A) \)
    \item \( A \subseteq f^{-1}(B) \)
    \item \( f^{-1}(B) \subseteq A \)
\end{enumerate}

\subsection*{Lösung und Begründung}

\begin{itemize}
    \item[(i)] \textbf{Wahr.}
        Wenn \( x \in A \) gilt, so ist \( f(x) \) per Definition ein Element von \( f(A) = B \).
        Daher gilt \( x \in A \Rightarrow f(x) \in B \).

    \item[(ii)] \textbf{Falsch.}
        Aus \( f(x) \in B \) folgt nur, dass es ein \( a \in A \) mit \( f(a) = f(x) \) gibt.
        Das bedeutet nicht zwingend \( x \in A \), außer \( f \) ist injektiv.
        \[
            \text{Beispiel: } X = \{1,2\},\; A = \{1\},\; f(1) = f(2) = 0.
        \]
        Dann ist \( B = \{0\} \), und \( f(2) \in B \), aber \( 2 \notin A \).

    \item[(iii)] \textbf{Wahr.}
        Es gilt \( f^{-1}(B) = \{ x \in X : f(x) \in B \} \).
        Für alle \( a \in A \) gilt \( f(a) \in B \), also \( a \in f^{-1}(B) \).
        Damit folgt \( A \subseteq f^{-1}(B) \).

    \item[(iv)] \textbf{Falsch.}
        Im obigen Gegenbeispiel ist \( f^{-1}(B) = \{1,2\} \not\subseteq A = \{1\} \).
\end{itemize}

\subsection*{Bemerkung}
Die Aussagen (i) und (iii) sind logisch äquivalent (punktweise bzw. mengenweise formuliert).
Die Aussagen (ii) und (iv) gelten nur, falls \( f \) injektiv ist.

\end{document}
