\documentclass{article}

\usepackage[utf8]{inputenc}
\usepackage[T1]{fontenc}
\usepackage[ngerman]{babel}
\usepackage{amsmath, amssymb}
\usepackage{xcolor}
\usepackage{enumitem}
\usepackage{fancyhdr}

% Kopf- und Fußzeilen
\pagestyle{fancy}
\fancyhead{}
\fancyfoot{}
\fancyhead[R]{\thepage}
\renewcommand{\headrulewidth}{0.4pt}

\begin{document}

\section*{Aufgabe 1:e}
Seien \(X\) und \(Y\) Mengen und \(f\colon X\to Y\) und \(g\colon Y\to X\) Abbildungen. Zeigen Sie:
\begin{enumerate}[label=(\roman*)]
    \item Ist \(g\) Linksinverse von \(f\), d.\,h.\ \(g\circ f=\mathrm{id}_X\), so ist \(f\) injektiv.
    \item Ist \(g\) Rechtsinverse von \(f\), d.\,h.\ \(f\circ g=\mathrm{id}_Y\), so ist \(f\) surjektiv.
\end{enumerate}

\subsection*{Beweis}

\textbf{(i) $g\circ f=\mathrm{id}_X \Rightarrow f$ injektiv.}

Seien \(x,x'\in X\) mit \(f(x)=f(x')\). Wenden wir \(g\) auf beide Seiten an, so erhalten wir
\[
    g(f(x))=g(f(x')).
\]
Da \(g\circ f=\mathrm{id}_X\) gilt, ist \(g(f(x))=x\) und \(g(f(x'))=x'\). Somit folgt \(x=x'\). Daher ist \(f\) injektiv.

\bigskip

\textbf{(ii) $f\circ g=\mathrm{id}_Y \Rightarrow f$ surjektiv.}

Sei \(y\in Y\) beliebig. Setze \(x:=g(y)\in X\). Dann gilt
\[
    f(x)=f(g(y))=(f\circ g)(y)=\mathrm{id}_Y(y)=y.
\]
Also existiert für jedes \(y\in Y\) ein \(x\in X\) mit \(f(x)=y\). Daher ist \(f\) surjektiv.

\bigskip

\noindent Damit sind beide Aussagen gezeigt.

\end{document}
